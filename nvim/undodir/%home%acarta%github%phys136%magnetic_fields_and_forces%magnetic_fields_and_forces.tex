Vim�UnDo����W�h�S`�-<�c��@,��>Enx%2�\title{Circuits}[�y	_������V[�y�
�(\section*{Circuit Elements and Diagrams}
\begin{frame},  \frametitle{Circuit Elements and Diagrams}  \begin{itemize}  \itemN    Rather than drawing a literal picture of a circuit, we use a more abstract.    picture called a \textbf{circuit diagram}.  \itemP    A circuit diagram is a \textbf{logical} picture of what's connected to what.  \itemI    The actual circuit may \textbf{look} quite different from the circuit=    diagram, but it will have the same logic and connections.  \end{itemize}\end{frame}
\begin{frame},  \frametitle{Circuit Elements and Diagrams}  \begin{itemize}  \item ,    Basic symbols used for circuit diagrams:  \end{itemize}  \begin{figure}    \centering<    \includegraphics[width=\textwidth]{fig/23_02_Figure.jpg}  \end{figure}\end{frame}
\begin{frame},  \frametitle{Circuit Elements and Diagrams}  \begin{columns}!    \begin{column}{0.5\textwidth}      \begin{itemize}      \item N        The battery's emf $\cale$, the resistance $R$, and the capacitance $C$A        of the capacitor are written beside the circuit elements.      \itemE        The wires, which may bend and curve in practice, are shown as?        straight-line connections between the circuit elements.      \end{itemize}    \end{column}!    \begin{column}{0.5\textwidth}      \begin{figure}[h]        \centeringC        \includegraphics[width=0.7\textwidth]{fig/23_01_Figure.jpg}      \end{figure}      \begin{figure}[h]        \centeringC        \includegraphics[width=0.7\textwidth]{fig/23_03_Figure.jpg}      \end{figure}    \end{column}  \end{columns}\end{frame}\section*{Kirchoff's Laws}
\begin{frame}&  \frametitle{Kirchoff's Junction Law}  \begin{columns}!    \begin{column}{0.5\textwidth}      \begin{itemize}      \item X        Since charge and current must be conserved the total current entering a junction:        must equal the total current leaving the junction.      \item:        This is known as \textbf{Kirchoff's junction law}:        \begin{equation*}          \label{eq:junc_law}$          \sum I_{in} = \sum I_{out}        \end{equation*}      \end{itemize}    \end{column}!    \begin{column}{0.5\textwidth}      \begin{figure}        \centeringC        \includegraphics[width=0.7\textwidth]{fig/23_04_Figure.jpg}      \end{figure}    \end{column}  \end{columns}\end{frame}
\begin{frame}"  \frametitle{Kirchoff's Loop Law}  \begin{itemize}  \item G    The gravitational potential energy of an object depends only on its>    position, not on the path it took to get to that position.  \item2    The same is true of electric potential energy.  \itemP    If a charged particle moves around a closed loop and returns to its startingH    point, there is no net change in its electric potential energy, i.e.    \begin{equation*}      \Delta U_e = 0.    \end{equation*}  \itemO    Since, by definition, $V = U_e / q$, this means that \textbf{the net changeI      in electric potential around any loop or closed path must be zero}.  \end{itemize}\end{frame}
\begin{frame}"  \frametitle{Kirchoff's Loop Law}  \begin{columns}!    \begin{column}{0.5\textwidth}    \begin{figure}      \centering?      \includegraphics[width=\textwidth]{fig/23_05_FigureA.jpg}    \end{figure}    \end{column}!    \begin{column}{0.5\textwidth}    \begin{figure}      \centering?      \includegraphics[width=\textwidth]{fig/23_05_FigureB.jpg}    \end{figure}    \end{column}  \end{columns}\end{frame}
\begin{frame}"  \frametitle{Kirchoff's Loop Law}  \begin{columns}!    \begin{column}{0.5\textwidth}      \begin{itemize}      \item L        For any circuit, the sum of all the potential differences around the0        loop formed by the circuit must be zero.      \item4        This is called \textbf{Kirchoff's loop law}:        \begin{equation*}2          \Delta V_{loop} = \sum_i \Delta V_i = 0,        \end{equation*}W        where $\Delta V_i$ is the potential difference of the $i^{\text{th}}$ component        of the loop.      \end{itemize}    \end{column}!    \begin{column}{0.5\textwidth}      \begin{figure}        \centeringA        \includegraphics[width=\textwidth]{fig/23_05_FigureC.jpg}      \end{figure}    \end{column}  \end{columns}\end{frame}
\begin{frame}"  \frametitle{Kirchoff's Loop Law}  \begin{itemize}  \item M    For a battery, $\Delta V$ can be positive or negative. The current can goL    through a battery in the ``wrong'' (i.e. positive-to-negative) direction@    when it's forced to do so by other higher voltage batteries.  \itemL    For a resistor, $\Delta V$ is always positive because the potential in aC    resistor \textbf{decreases} along the direction of the current.  \itemO    Because the potential across a resistor always decreases, we often speak of0    the \textbf{voltage drop} across a resistor.  \end{itemize}\end{frame}
\begin{frame}  \frametitle{Kirchoff's Laws}  \begin{itemize}  \item N    The most basic electric circuit consists of a single resistor connected to#    the two terminals of a battery.  \itemS    There are no junctions, so the current is the same in all parts of the circuit.  \end{itemize}  \begin{figure}[h]    \centering?    \includegraphics[width=0.5\textwidth]{fig/23_06_Figure.jpg}E    \caption{The basic circuit of a resistor connected to a battery.}  \end{figure}\end{frame}
\begin{frame}  \frametitle{Kirchoff's Laws}  \begin{itemize}  \item "    Using the Kirchoff's loop law,    \begin{equation*}K      \Delta V_{loop} = \sum_i \Delta V_i = \Delta V_{bat} + \Delta V_R = 0    \end{equation*}  \end{itemize}  \begin{figure}[h]    \centering?    \includegraphics[width=0.5\textwidth]{fig/23_06_Figure.jpg}E    \caption{The basic circuit of a resistor connected to a battery.}  \end{figure}\end{frame}
\begin{frame}  \frametitle{Kirchoff's Laws}  \begin{itemize}  \item I    As a charge passes through the battery from $+$ to $-$, its potential    increases by $\cale$.  \item#    Thus, $\Delta V_{bat} = \cale$.  \end{itemize}  \begin{figure}[h]    \centering?    \includegraphics[width=0.5\textwidth]{fig/23_06_Figure.jpg}E    \caption{The basic circuit of a resistor connected to a battery.}  \end{figure}\end{frame}
\begin{frame}  \frametitle{Kirchoff's Laws}  \begin{itemize}  \item N    For the resistor, Ohm's law tells us that $\abs{\Delta V} = IR$, but is it    positive or negative?  \itemJ    The potential of a resistor \textbf{decreases} in the direction of the'    current, so $\Delta V_{bat} = - IR$  \end{itemize}  \begin{figure}[h]    \centering?    \includegraphics[width=0.5\textwidth]{fig/23_06_Figure.jpg}E    \caption{The basic circuit of a resistor connected to a battery.}  \end{figure}\end{frame}
\begin{frame}  \frametitle{Kirchoff's Laws}  \begin{itemize}  \item *    Substituting into Kirchoff's loop law,    \begin{equation*}      \cale - IR = 0.    \end{equation*}  \item(    The current is then $I = \cale / R$.  \end{itemize}  \begin{figure}[h]    \centering?    \includegraphics[width=0.5\textwidth]{fig/23_07_Figure.jpg}  \end{figure}\end{frame}
\begin{frame}  \frametitle{Example}S  There is a current of $\SI{1.0}{\ampere}$ in this circuit. What is the resistance!  of the unknown circuit element?  \begin{figure}[h]    \centering6    \includegraphics[width=0.7\textwidth]{fig/ex1.jpg}  \end{figure}\end{frame}
\begin{frame}3  \frametitle{Example (Circuit with Two Batteries)}L  What is the current in the circuit below? What is the potential difference  across each resistor?  \begin{figure}[h]    \centering?    \includegraphics[width=0.8\textwidth]{fig/23_08_Figure.jpg}-    \caption{The circuit with two batteries.}  \end{figure}\end{frame}'\section*{Series and Parallel Circuits}
\begin{frame}+  \frametitle{Series and Parallel Circuits}  \begin{itemize}  \item K    There are two possible ways that you can connect the following circuit.  \itemF    \textbf{Series} and \textbf{parallel} circuits have very different    properties.  \itemL    We say two circuit elements are connected in \textbf{series} if they are3    connected directly with no junction in between.  \end{itemize}  \begin{figure}[h]    \centering?    \includegraphics[width=0.9\textwidth]{fig/23_10_Figure.jpg}  \end{figure}\end{frame}
\begin{frame}  \frametitle{Series Resistors}  \begin{columns}!    \begin{column}{0.5\textwidth}      \begin{itemize}      \item H        The figure shows two resistors in series connected to a battery.      \item[        Because there are no junctions, the current $I$ must be the same in both resistors.      \end{itemize}    \end{column}!    \begin{column}{0.5\textwidth}      \begin{figure}[h]        \centeringD        \includegraphics[width=0.9\textwidth]{fig/23_13_FigureA.jpg}      \end{figure}    \end{column}  \end{columns}\end{frame}
\begin{frame}  \frametitle{Series Resistors}  \begin{columns}!    \begin{column}{0.5\textwidth}      \begin{itemize}      \item"        Using Kirchoff's loop law,        \begin{align*}B         \sum_i \Delta V_i &= \epsi + \Delta V_1 + \Delta V_2 = 0.        \end{align*}      \itemK        The voltage drops across the two resistors, in the direction of the        current $I$, are        \begin{align*}(          \Delta V_1 &= - I R_1 \aand \\          \Delta V_2 &= - I R_2        \end{align*}      \end{itemize}    \end{column}!    \begin{column}{0.5\textwidth}      \begin{figure}[h]        \centeringD        \includegraphics[width=0.9\textwidth]{fig/23_13_FigureA.jpg}      \end{figure}    \end{column}  \end{columns}\end{frame}
\begin{frame}  \frametitle{Series Resistors}  \begin{columns}!    \begin{column}{0.5\textwidth}      \begin{itemize}      \item.        Substituting into Kirchoff's loop law,        \begin{align*}1          0 &= \cale + \Delta V_1 + \Delta V_2 \\#          &= \cale - I R_1 - I R_2.        \end{align*}      \item        The current $I$ is then        \begin{equation*}&          I = \frac{\cale}{R_1 + R_2}.        \end{equation*}      \end{itemize}    \end{column}!    \begin{column}{0.5\textwidth}      \begin{figure}[h]        \centeringD        \includegraphics[width=0.9\textwidth]{fig/23_13_FigureA.jpg}      \end{figure}    \end{column}  \end{columns}\end{frame}
\begin{frame}  \frametitle{Series Resistors}  \begin{columns}!    \begin{column}{0.5\textwidth}      \begin{itemize}      \itemO        If we replace the two resistors with a single resistor having the valueO        $R_{\text{eq}} = R_1 + R_2$, the total potential difference across thisE        resistor is still $\cale$ because the potential difference is#        established by the battery.      \end{itemize}    \end{column}!    \begin{column}{0.5\textwidth}      \begin{figure}[h]        \centeringD        \includegraphics[width=0.9\textwidth]{fig/23_13_FigureB.jpg}      \end{figure}    \end{column}  \end{columns}\end{frame}
\begin{frame}  \frametitle{Series Resistors}  \begin{columns}!    \begin{column}{0.5\textwidth}      \begin{itemize}      \item5        The current in the single resistor circuit is        \begin{equation*}D          I = \frac{\cale}{R_{\text{eq}}} = \frac{\cale}{R_1 + R_2}.        \end{equation*}      \itemO        The single resistor is \textbf{equivalent} to the two series resistors,N        since the current and potential difference are the same in both cases.      \itemP        If we have $N$ resistors in series, their \textbf{equivalent resistance}
        is        \begin{equation*}2          R_{\text{eq}} = R_1 + R_2 + \dots + R_N.        \end{equation*}      \end{itemize}    \end{column}!    \begin{column}{0.5\textwidth}      \begin{figure}[h]        \centeringD        \includegraphics[width=0.9\textwidth]{fig/23_13_FigureB.jpg}      \end{figure}    \end{column}  \end{columns}\end{frame}
\begin{frame}0  \frametitle{Example (Series Resistor Circuit)}&  What is the current in this circuit?  \begin{figure}[h]    \centering?    \includegraphics[width=0.6\textwidth]{fig/23_14_Figure.jpg}(    \caption{A series resistor circuit.}  \end{figure}\end{frame}
\begin{frame}H  \frametitle{Example (Potential Difference of a String of Mini-Lights)}N  A string of Christmas-tree mini-lights consists of 50 bulbs wired in series.N  What is the potential difference across each bulb when the string is plugged"  into a $\SI{120}{\volt}$ outlet?\end{frame}
\begin{frame}!  \frametitle{Parallel Resistors}  \begin{itemize}  \item T    When two resistors in a circuit are connected at \textbf{both} ends, we say that,    they are connected in \textbf{parallel}.  \itemM    Since both resistors are connected directly to the battery, the potential&    difference across each is $\cale$.  \end{itemize}  \begin{figure}[h]    \centering@    \includegraphics[width=0.6\textwidth]{fig/23_20_FigureA.jpg}  \end{figure}\end{frame}
\begin{frame}!  \frametitle{Parallel Resistors}  \begin{itemize}  \item N    However, the current $I_{bat}$ from the battery splits into currents $I_1$)    and $I_2$ at the top of the junction.  \end{itemize}  \begin{figure}[h]    \centering@    \includegraphics[width=0.6\textwidth]{fig/23_20_FigureB.jpg}  \end{figure}\end{frame}
\begin{frame}!  \frametitle{Parallel Resistors}  \begin{itemize}  \item@    According to Kirchoff's junction law, $I_{bat} = I_1 + I_2$.  \item    Applying Ohm's law,    \begin{equation*}�      I_{bat} = \frac{\Delta V_1}{R_1} + \frac{\Delta V_2}{R_2} = \frac{\cale}{R_1} + \frac{\cale}{R_2} = \cale \p{\frac{1}{R_1} + \frac{1}{R_2}}    \end{equation*}  \end{itemize}  \begin{figure}[h]    \centering@    \includegraphics[width=0.6\textwidth]{fig/23_20_FigureB.jpg}  \end{figure}\end{frame}
\begin{frame}!  \frametitle{Parallel Resistors}  \begin{itemize}  \item%    What's the equivalent resistance?  \itemC    To be equivalent, we need $\Delta V = \cale$ and $I = I_{bat}$.  \itemN    A resistor with this potential difference and current must have resistance    \begin{equation*}j      R_{\text{eq}} = \frac{\Delta V}{I} = \frac{\cale}{I_{bat}} = \frac{1}{\frac{1}{R_1} + \frac{1}{R_2}}    \end{equation*}  \end{itemize}  \begin{figure}[h]    \centering@    \includegraphics[width=0.6\textwidth]{fig/23_20_FigureC.jpg}  \end{figure}\end{frame}
\begin{frame}!  \frametitle{Parallel Resistors}  \begin{itemize}  \itemD    The \textbf{equivalent resistance} of $N$ resistors connected in    \textbf{parallel} is    \begin{equation*}�      R_{\text{eq}} = \frac{1}{\frac{1}{R_1} + \frac{1}{R_2} + \cdots + \frac{1}{R_N}} = \p{\frac{1}{R_1} + \cdots + \frac{1}{R_N}}^{_-1}.    \end{equation*}  \end{itemize}  \begin{figure}[h]    \centering@    \includegraphics[width=0.6\textwidth]{fig/23_20_FigureC.jpg}  \end{figure}\end{frame}
\begin{frame}?  \frametitle{Example (Current in a Parallel Resistor Circuit)}P  The three resistors in the figure are connected to a $\SI{12}{\volt}$ battery.*  What current is provided by the battery?  \begin{figure}[h]    \centering?    \includegraphics[width=0.6\textwidth]{fig/23_22_Figure.jpg}*    \caption{A parallel resistor circuit.}  \end{figure}\end{frame}
\begin{frame}!  \frametitle{Parallel Resistors}  \begin{itemize}  \item A    It would seem that more resistors $\implies$ more resistance.  \itemN    This is \textbf{true} for resistors in \textbf{series}, but \textbf{false}'    for resistors in \textbf{parallel}.  \itemH    Parallel resistors provide more pathways for charge to pass through.  \itemP    \textbf{The equivalent resistance of several resistors in parallel is always2      less than any single resistor in the group}.  \itemN    An analogy is driving in heavy traffic. If there is an alternate route for3    cars to travel, more cars will be able to flow.  \end{itemize}\end{frame}
\begin{frame}+  \frametitle{Example (Parallel Resistors)}>  What is the current supplied by the battery in this circuit?  \begin{figure}[h]    \centering6    \includegraphics[width=0.7\textwidth]{fig/ex2.jpg}  \end{figure}\end{frame}
\begin{frame}  \frametitle{Example (Heater)}K  A resistor connected to a power supply can work as a heater. Which of the1  following two circuits will provide more power?  \begin{figure}[h]    \centering6    \includegraphics[width=0.9\textwidth]{fig/ex3.jpg}  \end{figure}\end{frame} \section*{More Complex Circuits}
\begin{frame}4  \frametitle{Example (Analyzing a Complex Circuit)}O  What is the current supplied by the battery in the following circuit? What isI  the current through each resistor in the circuit? What is the potential"  difference across each resistor?  \begin{figure}[h]    \centering6    \includegraphics[width=0.7\textwidth]{fig/ex4.jpg}  \end{figure}\end{frame}
\begin{frame}C  \frametitle{Example (Equivalent Resistance of a Complex Circuit)}4  What is the equivalent resistance of this circuit?  \begin{figure}[h]    \centering6    \includegraphics[width=0.7\textwidth]{fig/ex5.jpg}  \end{figure}\end{frame},\section*{Capacitors in Parallel and Series}
\begin{frame}0  \frametitle{Capacitors in Parallel and Series}  \begin{columns}!    \begin{column}{0.5\textwidth}      \begin{itemize}      \item H        A \textbf{capacitor} is a circuit element made of two conductors)        separated by an insulating layer.      \itemO        When the capacitor is connected to the battery, charge will flow to theR        capacitor, increasing its potential difference until $\Delta V_c = \cale$.      \itemM        Once the capacitor is fully charged (and $\Delta V_c = \cale$), there#        will be no further current.      \end{itemize}    \end{column}!    \begin{column}{0.5\textwidth}      \begin{figure}[h]        \centering@        \includegraphics[width=\textwidth]{fig/23_31_Figure.jpg}      \end{figure}    \end{column}  \end{columns}\end{frame}
\begin{frame}0  \frametitle{Capacitors in Parallel and Series}  \begin{itemize}  \item L    Like resistors, capacitors in parallel or series can be represented by a+    single \textbf{equivalent capacitance}.  \end{itemize}  \begin{figure}[h]    \centering?    \includegraphics[width=0.6\textwidth]{fig/23_32_Figure.jpg}  \end{figure}\end{frame}
\begin{frame}%  \frametitle{Capacitors in Parallel}  \begin{figure}[h]    \centering@    \includegraphics[width=0.5\textwidth]{fig/23_33_FigureA.jpg}  \end{figure}  \begin{itemize}  \item 1    The total charge $Q$ on the two capacitors is    \begin{equation*}P      Q = Q_1 + Q_2 = C_1 \Delta V_c + C_2 \Delta V_c = \p{C_1 + C_2} \Delta V_c    \end{equation*}  \end{itemize}\end{frame}
\begin{frame}%  \frametitle{Capacitors in Parallel}  \begin{itemize}  \item M    We can replace two capacitors in \textbf{parallel} by a single equivalent     capacitance $C_{\text{eq}}$.  \itemQ    The equivalent capacitance has the same potential difference $\Delta V_c$ but+    a greater charge than either capacitor.  \end{itemize}  \begin{figure}[h]    \centering@    \includegraphics[width=0.5\textwidth]{fig/23_33_FigureB.jpg}  \end{figure}\end{frame}
\begin{frame}%  \frametitle{Capacitors in Parallel}  \begin{itemize}  \item O    If $N$ capacitors are in \textbf{parallel}, their equivalent capacitance is    \begin{equation*}-      C_{\text{eq}} = C_1 + C_2 + \cdots C_N.    \end{equation*}  \end{itemize}\end{frame}
\begin{frame}#  \frametitle{Capacitors in Series}  \begin{itemize}  \item A    The potential differences across two capacitors in series are    \begin{equation*}B      \Delta V_1 = \frac{Q}{C_1} \aand \Delta V_2 = \frac{Q}{C_2}.    \end{equation*}  \item<    The total potential difference across both capacitors is    \begin{equation*}+      \Delta V_c = \Delta V_1 + \Delta V_2.    \end{equation*}  \end{itemize}  \begin{figure}[h]    \centering@    \includegraphics[width=0.6\textwidth]{fig/23_34_FigureA.jpg}  \end{figure}\end{frame}
\begin{frame}#  \frametitle{Capacitors in Series}  \begin{itemize}  \item N    If we replace the two capacitors with a single capacitor having charge $Q$/    and potential difference $\Delta V_C$, then    \begin{equation*}X      \frac{1}{C_{\text{eq}}} = \frac{\Delta V_c}{Q} = \frac{\Delta V_1 + \Delta V_2}{Q}3      = \frac{\Delta V_1}{Q} + \frac{\Delta V_2}{Q}&      = \frac{1}{C_1} + \frac{1}{C_2}.    \end{equation*}  \end{itemize}  \begin{figure}[h]    \centering@    \includegraphics[width=0.6\textwidth]{fig/23_34_FigureB.jpg}  \end{figure}\end{frame}
\begin{frame}#  \frametitle{Capacitors in Series}  \begin{itemize}  \item D    If $N$ capacitors are in series, their equivalent capacitance is    \begin{equation*}V      C_{\text{eq}} = \frac{1}{\frac{1}{C_1} + \frac{1}{C_2} + \cdots + \frac{1}{C_N}}H      = \p{\frac{1}{C_1} + \frac{1}{C_2} + \cdots + \frac{1}{C_N}}^{-1}.    \end{equation*}  \itemM    For capacitors in series, the equivalent capacitance is less than that of    the individual capacitors.  \end{itemize}\end{frame}
\begin{frame}6  \frametitle{Example (Analyzing a Capacitor Circuit)}#  \begin{enumerate}[label=(\alph*)]  \item K    Find the equivalent capacitance of the combination of capacitors in the    circuit.  \itemN    What charge flows through the battery as the capacitors are being charged?  \end{enumerate}  \begin{figure}[h]    \centering?    \includegraphics[width=0.6\textwidth]{fig/23_35_Figure.jpg}  \end{figure}\end{frame}\section*{RC Circuits}
\begin{frame}  \frametitle{$RC$ Circuits}  \begin{itemize}  \item 3    $RC$ circuits contain resistors and capacitors.  \item7    In $RC$ circuits, the current $I$ varies with time.  \itemK    The values of the resistance $R$ and capacitance $C$ in an $RC$ circuitE    determine the time it takes the capacitor to charge or discharge.  \end{itemize}\end{frame}
\begin{frame}&  \frametitle{Discharging a Capacitor}  \begin{itemize}  \item K    Below is an $RC$ circuit consisting of a \textbf{charged capacitor}, anK    \textbf{open switch}, and a \textbf{resistor} before the switch closes.  \item#    The switch will close at $t=0$.  \end{itemize}  \begin{figure}[h]    \centering@    \includegraphics[width=0.4\textwidth]{fig/23_37_FigureA.jpg}  \end{figure}\end{frame}
\begin{frame}&  \frametitle{Discharging a Capacitor}  \begin{columns}!    \begin{column}{0.5\textwidth}      \begin{itemize}      \item J        After the switch closes, the capacitor voltage is still $\p{\DeltaK          V_c}_0$ because the capacitor hasn't had time to lose any charge.      \itemN        However, there's now a current $I_0$ in the circuit that's starting to         discharge the capacitor.      \end{itemize}    \end{column}!    \begin{column}{0.5\textwidth}      \begin{figure}[h]        \centeringA        \includegraphics[width=\textwidth]{fig/23_37_FigureB.jpg}      \end{figure}    \end{column}  \end{columns}\end{frame}
\begin{frame}&  \frametitle{Discharging a Capacitor}  \begin{columns}!    \begin{column}{0.5\textwidth}      \begin{itemize}      \item +        The initial potential difference is        \begin{equation*}+          \p{\Delta V_c}_0 = \frac{Q_0}{C}.        \end{equation*}      \item        The initial current is        \begin{equation*}+          I_0 = \frac{\p{\Delta V_c}_0}{R}.        \end{equation*}      \end{itemize}    \end{column}!    \begin{column}{0.5\textwidth}      \begin{figure}[h]        \centeringA        \includegraphics[width=\textwidth]{fig/23_37_FigureB.jpg}      \end{figure}    \end{column}  \end{columns}\end{frame}
\begin{frame}&  \frametitle{Discharging a Capacitor}  \begin{columns}!    \begin{column}{0.5\textwidth}      \begin{itemize}      \item G        After some time, both the charge on the capacitor (and thus the=        potential difference) and the current have decreased.      \itemM        When the capacitor voltage has decreased to $\Delta V_c$, the current        has decreased to        \begin{equation*}#          I = \frac{\Delta V_c}{R}.        \end{equation*}      \itemI        The current and the voltage decrease until the capacitor is fully+        discharged and the current is zero.      \end{itemize}    \end{column}!    \begin{column}{0.5\textwidth}      \begin{figure}[h]        \centeringA        \includegraphics[width=\textwidth]{fig/23_37_FigureC.jpg}      \end{figure}    \end{column}  \end{columns}\end{frame}
\begin{frame}&  \frametitle{Discharging a Capacitor}  \begin{columns}!    \begin{column}{0.5\textwidth}      \begin{itemize}      \item P        The current and capacitor voltage decay to zero after the switch closes,"        but \textbf{not} linearly.      \end{itemize}    \end{column}!    \begin{column}{0.5\textwidth}      \begin{figure}[h]        \centering@        \includegraphics[width=\textwidth]{fig/23_38_Figure.jpg}      \end{figure}    \end{column}  \end{columns}\end{frame}
\begin{frame}&  \frametitle{Discharging a Capacitor}  \begin{itemize}  \item N    When discharging a capacitor, the current and potential difference at time    $t$ are    \begin{align*}"      I(t) &= I_0 e^{-t / \tau} \\7      \Delta V_c (t) &= \p{\Delta V_c}_0 e^{-t / \tau},    \end{align*}C    where $\tau = RC$ is the \textbf{time constant} of the circuit.  \itemO    The time constant is a characteristic time for a circuit. The larger $\tau$    is, the slower the decay.  \end{itemize}  \end{frame}
\begin{frame}&  \frametitle{Discharging a Capacitor}  \begin{itemize}  \item N    A large resistance opposes the flow of charge, so increasing $R$ increases    the decay time.  \itemM    A larger capacitance stores more charge, so increasing $C$ also increases    the decay time.  \end{itemize}  \begin{figure}[h]    \centering?    \includegraphics[width=0.5\textwidth]{fig/23_39_Figure.jpg}  \end{figure}\end{frame}
\begin{frame}=  \frametitle{Example (Finding the Current in an RC Circuit)}I  The switch in this circuit has been in position for a long time, so theK  capacitor is fully charged. The switch is changed to position $b$ at time  $t=0$.#  \begin{enumerate}[label=(\alph*)]  \itemN    What is the current in the circuit immediately after the switch is closed?  \itemA    What is the current in the circuit $\SI{25}{\micro\s}$ later?  \end{enumerate}  \begin{figure}[h]    \centering?    \includegraphics[width=0.7\textwidth]{fig/23_40_Figure.jpg}  \end{figure}\end{frame}
\begin{frame}#  \frametitle{Charging a Capacitor}  \begin{columns}!    \begin{column}{0.5\textwidth}      \begin{itemize}      \item J        In a circuit that \textbf{charges} a capacitor, once the switch isO        closed, the potential difference of the battery causes a current in the4        circuit, and the capacitor begins to charge.      \itemI        As the capacitor charges, it develops a potential difference that6        opposes the current, so the current decreases.      \item;        This means the rate of charging decreases, as well.      \item9        The capacitor charges until $\Delta V_c = \cale$.      \end{itemize}    \end{column}!    \begin{column}{0.5\textwidth}      \begin{figure}[h]        \centeringA        \includegraphics[width=\textwidth]{fig/23_41_FigureA.jpg}      \end{figure}    \end{column}  \end{columns}\end{frame}
\begin{frame}#  \frametitle{Charging a Capacitor}  \begin{columns}!    \begin{column}{0.5\textwidth}      \begin{itemize}      \item a        When charging a capacitor, the capacitor current and potential difference at time $t$ are        \begin{align*}$          I(t) &= I_0 e^{-t/\tau} \\7          \Delta V_c (t) &= \cale \p{1 - e^{-t /\tau}}>        \end{align*}      \end{itemize}    \end{column}!    \begin{column}{0.5\textwidth}      \begin{figure}[h]        \centeringD        \includegraphics[width=0.8\textwidth]{fig/23_41_FigureB.jpg}      \end{figure}    \end{column}  \end{columns}\end{frame}5�_�����V[�y�\title{Circuits}5�_�����V[�y�\title{}5��